\documentclass[11pt]{beamer}
\usetheme{CambridgeUS}
\usepackage[utf8]{inputenc}
\usepackage{amsmath}
\usepackage{amsfonts}
\usepackage{amssymb}
\usepackage{array}
\usepackage{multirow}
\usepackage{tabularx, booktabs}


\author{Adam Sudit}
\title{Bank Distress and Unemployment During the Great Depression:
A Plant-Level Approach}
\institute{The University of Chicago} 

\begin{document}

\begin{frame}
\titlepage
\end{frame}

\begin{frame}{Financial Frictions During Economic Crisis}

\begin{itemize}
\item Financial crises wreak havoc on the real economy. But through what mechanisms?
\begin{itemize}
\item Monetary channels -- Decrease in Supply of Money (Freidman and Schwartz, 1963)
\item Cost of Credit Intermediation (Bernanke, 1983)
\item Lending Constraints Negatively Affect Firm Investment and Output (Kashyap et. al, 1993 and Bernanke et. al, 1996)
\end{itemize}
\end{itemize}
\end{frame}

\begin{frame}{Disentangling Supply from Demand Effects}
\begin{itemize}
\item Do banks supply less credit because they're distressed, or because investment opportunities are poor?
\begin{itemize}
\item Use natural experiments to introduce exogenous shocks to credit supply (Ziebarth, 2013; Peek and Rosengren, 2000; Khwaja and Mian, 2008)
\item Use pre-crisis balance sheet health or lending relationships to proxy financial frictions (Benmelech et. al, 2017; Chodorow-Reich, 2014)
\end{itemize}
\end{itemize}
\end{frame}

\begin{frame}{Where We're Going}
\begin{itemize}
\item How does restrictions in credit access, as proxied by bank distress, affect employment outcomes in the manufacturing sector? (Adversely)
\item Do employment effects vary by industry, and if so, why?
\item Use better dataset to increase identifying variation, and add more industry heterogeneity to sample
\end{itemize}
\end{frame}

\begin{frame}{Historical Background}
\begin{itemize}
\item Stock market crash on October 1929, but first banking panic on November 1930, followed by another on March 1931
\item Britain leaves gold standard during September 1931, leading to further deterioration in bank balance sheets
\item By March 1933, Federal Reserve System suspends open market transactions
\item Bank holiday declared on March 6, 1933
\end{itemize}
\end{frame}

\begin{frame}{Structure of U.S. Banking System}
\begin{itemize}
\item Highly fragmented between local, state-chartered banks, and national banks
\item Interstate and within-state branching was uncommon and often non-existent
\item Unit banks prevalent
\end{itemize}
\end{frame}

\begin{frame}{State Legislation Restricting Bank Branching}
\begin{table}  \tiny{
\begin{tabular}{lcc}
State &1910 & 1929 \\  
\hline \hline \\
Alabama & & X \\
Alaska & & \\
Arizona & & * \\
Arkansas & & X \\
California & * & * \\
Colorado & X & X \\
Connecticut & X & X \\
Delaware & * & *\\
Florida & * & X \\
Georgia & * & O \\
Hawaii & & \\
Idaho & & X \\
Illinois & & X \\
Indiana & & X \\
Iowa & & X\\
Kansas & X \\
Kentucky & & \\
Louisiana & O & O \\
Maine & O & O \\
Maryland & & * \\
Massachusetts & X & O \\
Michigan &  & \\
Minnesotsa & & X \\
Mississippi & X & O \\
Missouri & X & X \\
Montana &  & X \\ 
\hline                                                                                                                             

\end{tabular}
}
\caption{\tiny{X: Only Unit Banking is Allowed; O: State Bank Branching Allowed, but Restrained; *: Branching Permitted Throughout State}}
\end{table}
\end{frame}


%\begin{frame}{State Legislation Restricting Bank Branching}
%\begin{figure}
%\centering{
%\includegraphics[scale=.325]{"/Users/Adam/Research/BA_Thesis/Spring_Presentation/Bank laws better".pdf}
%}
%\end{figure}
%\end{frame}

\begin{frame}{State Legislation Restricting Bank Branching, Continued}
\begin{figure}
\centering{
\includegraphics[scale=.325]{"/Users/Adam/Research/BA_Thesis/Spring_Presentation/Bank laws 2 better".pdf}
} 
\end{figure}
\end{frame}

\begin{frame}{Why the Historical Background Matters}
\begin{itemize}
\item Small, unit banks were an important source of financing to local firms
\item Bank-specificc sources of distress were often isolated to the regional level, due restrictions on bank branching
\item Highly relevant to the identification strategy
\end{itemize}
\end{frame}

\begin{frame}{Data}
\begin{itemize}
\item Manufacturing plant data on the county level, including total value added and labor (Census)
\item Bank suspensions, total number of banks, and deposits on the county level (FDIC)
\item Change in Mortgage Debt per Acre, 1920-1910 (Rajan and Ramcharan 2015)
\item External Finance Dependence, by Industry (Rajan and Zingales, 1998; Nanda and Nicholas, 2014)
\end{itemize}
\end{frame}

\begin{frame}{Bank Distress}
\begin{itemize}
\item Variable of Interest is Bank Distress, a fixed county characteristic:
\[
BankDistress_c = \frac{\sum_{t=1930}^{1935} BankSuspensions_{c, t}} {Banks_{c, 1929}}
\]
\item FDIC data includes both banks within and outside of the Federal Reserve System, so bank distress measure is complete across the U.S.
\end{itemize}
\end{frame}





\end{document}