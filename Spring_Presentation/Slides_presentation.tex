\documentclass[11pt]{beamer}
\usetheme{CambridgeUS}
\usepackage[utf8]{inputenc}
\usepackage{amsmath}
\usepackage{amsfonts}
\usepackage{amssymb}
\usepackage{array}
\usepackage{multirow}
\usepackage{tabularx, booktabs}


\author{Adam Sudit}
\title{Bank Distress and Unemployment During the Great Depression:
A Plant-Level Approach}
\institute{The University of Chicago} 

\begin{document}

\begin{frame}
\titlepage
\end{frame}

\begin{frame}{Financial Frictions During Economic Crisis}

\begin{itemize}
\item Financial crises wreak havoc on the real economy. But through what mechanisms?
\begin{itemize}
\item Monetary channels -- Decrease in Supply of Money (Freidman and Schwartz, 1963)
\item Cost of Credit Intermediation (Bernanke, 1983)
\item Lending Constraints Negatively Affect Firm Investment and Output (Kashyap et. al, 1993 and Bernanke et. al, 1996)
\end{itemize}
\end{itemize}
\end{frame}

\begin{frame}{Disentangling Supply from Demand Effects}
\begin{itemize}
\item Do banks supply less credit because they're distressed, or because investment opportunities are poor?
\begin{itemize}
\item Use natural experiments to introduce exogenous shocks to credit supply (Ziebarth, 2013; Peek and Rosengren, 2000; Khwaja and Mian, 2008)
\item Use pre-crisis balance sheet health or lending relationships to proxy financial frictions (Benmelech et. al, 2017; Chodorow-Reich, 2014)
\end{itemize}
\end{itemize}
\end{frame}

\begin{frame}{Where We're Going}
\begin{itemize}
\item How does restrictions in credit access, as proxied by bank distress, affect employment outcomes in the manufacturing sector? (Adversely)
\item Do employment effects vary by industry, and if so, why?
\item Use better dataset to increase identifying variation, and add more industry heterogeneity to sample
\end{itemize}
\end{frame}

\begin{frame}{Historical Background}
\begin{itemize}
\item Stock market crash on October 1929, but first banking panic on November 1930, followed by another on March 1931
\item Britain leaves gold standard during September 1931, leading to further deterioration in bank balance sheets
\item By March 1933, Federal Reserve System suspends open market transactions
\item Bank holiday declared on March 6, 1933
\end{itemize}
\end{frame}

\begin{frame}{Structure of U.S. Banking System}
\begin{itemize}
\item Highly fragmented between local, state-chartered banks, and national banks
\item Interstate and within-state branching was uncommon and often non-existent
\item Unit banks prevalent
\end{itemize}
\end{frame}

\begin{frame}{State Legislation Restricting Bank Branching}
\begin{table}  \tiny{
\begin{tabular}{lcc}
State &1910 & 1929 \\  
\hline \hline \\
Alabama & & X \\
Alaska & & \\
Arizona & & * \\
Arkansas & & X \\
California & * & * \\
Colorado & X & X \\
Connecticut & X & X \\
Delaware & * & *\\
Florida & * & X \\
Georgia & * & O \\
Hawaii & & \\
Idaho & & X \\
Illinois & & X \\
Indiana & & X \\
Iowa & & X\\
Kansas & X \\
Kentucky & & \\
Louisiana & O & O \\
Maine & O & O \\
Maryland & & * \\
Massachusetts & X & O \\
Michigan &  & \\
Minnesotsa & & X \\
Mississippi & X & O \\
Missouri & X & X \\
Montana &  & X \\ 
\hline                                                                                                                             

\end{tabular}
}
\caption{\tiny{X: Only Unit Banking is Allowed; O: State Bank Branching Allowed, but Restrained; *: Branching Permitted Throughout State}}
\end{table}
\end{frame}

\begin{frame}{State Legislation Restricting Bank Branching, Continued}
\begin{table}  \tiny{
\begin{tabular}{lcc}
State &1910 & 1929 \\  
\hline \hline \\
Nebraska & & X \\
Nevada & X & X \\
New Hampshire &  & \\
New Jersey & & O \\
New Mexico & & X \\
New York & O & O \\
North Carolina & & * \\
North Dakota & & \\
Ohio & & O \\
Oklahoma & & \\
Oregon & * & X \\
Pennsylvania & X & O \\
Rhode Island & * & * \\
South Carolina & & * \\
South Dakota & & \\
Tennessee & * & O \\
Texas & X & X \\
Utah & & X \\
Vermont & & * \\
Virginia & & * \\
Washington & * & X \\
West Virginia & & X \\
Wisconsin & X & X \\
Wyoming & & \\
\hline                                                                                                                             

\end{tabular}
}
\caption{\tiny{X: Only Unit Banking is Allowed; O: State Bank Branching Allowed, but Restrained; *: Branching Permitted Throughout State}}
\end{table}
\end{frame}



\begin{frame}{Why the Historical Background Matters}
\begin{itemize}
\item Small, unit banks were an important source of financing to local firms
\item Bank-specificc sources of distress were often isolated to the regional level, due to restrictions on bank branching
\item My identification strategy hinges on geographic variation in bank distress on the county level
\end{itemize}
\end{frame}

\begin{frame}{Data}
\begin{itemize}
\item Manufacturing plant data on the county level, including total value added and labor (Census)
\item Bank suspensions, total number of banks, and deposits on the county level (FDIC)
\item Change in Mortgage Debt per Acre, 1920-1910 (Rajan and Ramcharan 2015)
\item External Finance Dependence, by Industry (Rajan and Zingales, 1998; Nanda and Nicholas, 2014)
\end{itemize}
\end{frame}

\begin{frame}{Summary Statistics}
\begin{table}  \caption{Summary Statistics} \tiny{
\begin{tabular}{lcc} 
\hline \hline \\
Observations & \multicolumn{1}{c}{45,213} \\
 Firms & \multicolumn{1}{c}{21,801} \\
 Industries & \multicolumn{1}{c}{22} \\
Years & \multicolumn{1}{c}{1929, 1931, 1933, 1935} \\
Counties & \multicolumn{1}{c}{2,198}\\
States & \multicolumn{1}{c}{47} \\
\hline                                                                                                                             

\end{tabular}
}
\end{table}
\end{frame}

\begin{frame}{Number of Bank Suspensions, 1929-1933}
\begin{table}  \tiny{
\begin{tabular}{lccccc} 
Region & 1929 & 1930 & 1931 & 1932 & 1933\\
\hline \hline \\
Central & \multicolumn{1}{c}{554} & \multicolumn{1}{c}{1216} & \multicolumn{1}{c}{910} & \multicolumn{1}{c}{2662} & \multicolumn{1}{c}{40}\\
 Mountain & \multicolumn{1}{c}{48} & \multicolumn{1}{c}{118} & \multicolumn{1}{c}{188} & \multicolumn{1}{c}{230} & \multicolumn{1}{c}{2}\\
 Mid-atlantic & \multicolumn{1}{c}{56} & \multicolumn{1}{c}{452} & \multicolumn{1}{c}{118} & \multicolumn{1}{c}{796} & \multicolumn{1}{c}{24}\\
 New England & \multicolumn{1}{c}{22} & \multicolumn{1}{c}{62} & \multicolumn{1}{c}{18} & \multicolumn{1}{c}{152} & \multicolumn{1}{c}{0}\\
 Northwestern & \multicolumn{1}{c}{828} & \multicolumn{1}{c}{1434} & \multicolumn{1}{c}{894} & \multicolumn{1}{c}{2262} & \multicolumn{1}{c}{26}\\
 Pacific & \multicolumn{1}{c}{24} & \multicolumn{1}{c}{106} & \multicolumn{1}{c}{174} & \multicolumn{1}{c}{298} & \multicolumn{1}{c}{0}\\
 South Atlantic & \multicolumn{1}{c}{402} & \multicolumn{1}{c}{384} & \multicolumn{1}{c}{176} & \multicolumn{1}{c}{506} & \multicolumn{1}{c}{10}\\
 South Central & \multicolumn{1}{c}{702} & \multicolumn{1}{c}{620} & \multicolumn{1}{c}{378} & \multicolumn{1}{c}{884} & \multicolumn{1}{c}{8}\\
\hline                                                                                                                             

\end{tabular}
}
\end{table}
\end{frame}

\begin{frame}{Number of Bank Suspensions as Percentage of Total Banks, 1929-1933}
\begin{table}  \tiny{
\begin{tabular}{lccccc} 
Region & 1929 & 1930 & 1931 & 1932 & 1933\\
\hline \hline \\
Central & \multicolumn{1}{c}{5.02} & \multicolumn{1}{c}{11.8} & \multicolumn{1}{c}{10.4} & \multicolumn{1}{c}{34.1} & \multicolumn{1}{c}{0.69}\\
 Mountain & \multicolumn{1}{c}{2.85} & \multicolumn{1}{c}{7.38} & \multicolumn{1}{c}{13.2} & \multicolumn{1}{c}{18.8} & \multicolumn{1}{c}{0.21}\\
 Mid-atlantic & \multicolumn{1}{c}{0.92} & \multicolumn{1}{c}{7.69} & \multicolumn{1}{c}{2.28} & \multicolumn{1}{c}{16.0} & \multicolumn{1}{c}{0.55}\\
 New England & \multicolumn{1}{c}{1.73} & \multicolumn{1}{c}{5.08} & \multicolumn{1}{c}{1.62} & \multicolumn{1}{c}{14.2} & \multicolumn{1}{c}{0}\\
 Northwestern & \multicolumn{1}{c}{6.54} & \multicolumn{1}{c}{12.5} & \multicolumn{1}{c}{9.13} & \multicolumn{1}{c}{25.6} & \multicolumn{1}{c}{0.33}\\
 Pacific & \multicolumn{1}{c}{1.22} & \multicolumn{1}{c}{5.61} & \multicolumn{1}{c}{10.3} & \multicolumn{1}{c}{20.5} & \multicolumn{1}{c}{0}\\
 South Atlantic & \multicolumn{1}{c}{9.78} & \multicolumn{1}{c}{10.7} & \multicolumn{1}{c}{5.50} & \multicolumn{1}{c}{16.5} & \multicolumn{1}{c}{0.38}\\
 South Central & \multicolumn{1}{c}{8.06} & \multicolumn{1}{c}{7.94} & \multicolumn{1}{c}{5.34} & \multicolumn{1}{c}{13.4} & \multicolumn{1}{c}{0.14}\\
\hline                                                                                                                             

\end{tabular}
}
\end{table}
\end{frame}

\begin{frame}{Bank Distress}
\begin{itemize}
\item Variable of Interest is Bank Distress, a fixed county characteristic:
\[
BankDistress_c = \frac{\sum_{t=1930}^{1935} BankSuspensions_{c, t}} {Banks_{c, 1929}}
\]
\item FDIC data includes both banks within and outside of the Federal Reserve System, so bank distress measure is complete across the U.S.
\item No direct data on bank loans to firms during the Great Depression exists
\end{itemize}
\end{frame}

\begin{frame}{Variation in Bank Distress}
\begin{figure}
\centering{
\includegraphics[scale=.375]{"/Users/Adam/Research/BA_Thesis/Graphs/Bank Distress Spatial Map".pdf}
}
\end{figure}
\end{frame}

\begin{frame}{Fixed Effects Model}
\begin{itemize}
\item The goal is to measure how county-level bank distress, which proxies restricted credit supply, affects the employment outcomes at a given firm in the same county
\item The identifying assumption is that manufacturing plants in a given county received loans from banks in the same county
\begin{itemize}
\item As discussed, the prevalence of small, unit banks and the highly fragmented nature of U.S banking meant that local banking was an important source of financing for firms
\item In the absence of branching, bank-specific shocks shouldn't spill over into other counties
\end{itemize}
\item Fixed effects are used to control for year, county, and industry-specific shocks
\item Firms are ``treated" with bank distress after 1929 using a dummy variable, since the first banking panic was in 1930
\end{itemize}
\end{frame}

\begin{frame}{Fixed Effects Model, continued}
\begin{itemize}
\item The fixed effects model is therefore:
\[
log(L_{jt}) = \beta_{0t} + \beta_{1t} Post\_1929XBank\_Distress_c + \gamma_t + \mu_c +\eta_i  + \epsilon_{ct}
\]
\item Where $L_{jt}$ is firm j's labor input in time t, $Post\_1929$ is a dummy variable for observations after 1929, and $\gamma_t$, $\mu_c$, $\eta_i$ are year, county, and industry fixed effects, respectively
\item The effect of interest is therefore the interaction effect between $Post\_1929$ and the measure of bank distress.
\end{itemize}
\end{frame}

\begin{frame}{Reverse Casuality and I.V.}
\begin{itemize}
\item Banks were more likely to fail or suspend operations in counties with greater disruptions in manufacturing activity
\begin{itemize}
\item Distressed manufacturing firms were more likely to default on bank loans, causing bank balance sheet health to deteriorate
\end{itemize}
\item Use change in mortgage debt per acre from 1910 to 1920 (on the county level) as an instrument for bank distress
\begin{itemize}
\item During the 1910's, the United States enjoyed a surge in land prices due to increased agricultural exports to Europe after the decimation of World War I left much of European farmland in ruin
\item Real estate bubble burst in 1920, causing national farm income to plummet
\item Change in mortgage debt per acre from 1920-1910 is a good predictor of bank suspensions between 1920-1929
\end{itemize}
\end{itemize}
\end{frame}


\begin{frame}{Reverse Casuality and I.V., continued}
\begin{itemize}
\item Banks in counties with increased mortgage debt per acre most likely had weaker balance sheets leading up to the Great Depression
\item The most plausible way in which mortgage debt per acre affected firm operations was by weakening the balance sheets of their lending banks.
\item Use 2SLS regression to instrument for bank distress. The first stage is:
\[
Post\_1929XBank\_Distress_c = \beta_{0t} + \beta_{1t} \Delta Mortgage\_Debt + \gamma_t + \mu_c +\eta_i  + \epsilon_{ct}
\]
\item Agricultural debt could be highly correlated with county-level shocks during the Great Depression, so the exclusion restriction is still an issue
\end{itemize}
\end{frame}


\begin{frame}{Main Results -- Fixed Effects}
\begin{table}  \tiny{
\begin{tabular}{lccc} 
\\[-1.8ex]\hline                                                                                     
 \hline \\[-1.8ex] 
log(Labor) & \multicolumn{1}{c}{(1)} & \multicolumn{1}{c}{(2)} & \multicolumn{1}{c}{(3)}\\    
 \hline \\                                         
 Post\_1929 x Bank Distress & $-0.181^{***}$ & $-0.143^{**}$ & $-0.003$ \\ 
  & (0.050) & (0.061) & (0.050) \\ 
\hline \\
Year Fixed Effects &  & \multicolumn{1}{c}{X} & \multicolumn{1}{c}{X} \\ 
County Fixed Effects &  & \multicolumn{1}{c}{X} & \multicolumn{1}{c}{X} \\ 
Industry Fixed Effects &  &  & \multicolumn{1}{c}{X} \\
Observations & \multicolumn{1}{c}{34,207} &   \multicolumn{1}{c}{34,207} & \multicolumn{1}{c}{34,207} \\
R-squared  &   \multicolumn{1}{c}{0.003975} & \multicolumn{1}{c}{0.2561} & \multicolumn{1}{c}{0.6921} \\
\hline 
\hline \\[-1.8ex] 
\textit{Notes:}  & \multicolumn{3}{r}{$^{*}$p$<$0.1; $^{**}$p$<$0.05; $^{***}$p$<$0.01}
\end{tabular}
}
\caption{\tiny{Robust standard errors are clustered at the county level. Sample restricted to firms open in 1929}}
\end{table}
\end{frame}

\begin{frame}{Main Results -- I.V.}
\begin{table}  \tiny{
\begin{tabular}{lcc} 
\\[-1.8ex]\hline
\hline \\[-1.8ex] 
\\[-1.8ex]  & \multicolumn{1}{c}{First Stage} & \multicolumn{1}{c}{Second Stage}\\                                     
\hline \\[-1.8ex]                                                                                   
Change in Mortgage  & $-0.020^{**}$ &  $-2.536^{**}$\\                                                                          
Debt Per Acre, 1920-1910 & (0.008) &  (1.272) \\ 
\hline \\[-1.8ex]
Year Fixed Effects & \multicolumn{1}{c}{X} & \multicolumn{1}{c}{X} \\ 
County Fixed Effects  & \multicolumn{1}{c}{X} & \multicolumn{1}{c}{X} \\ 
Industry Fixed Effects &  \multicolumn{1}{c}{X} & \multicolumn{1}{c}{X} \\
Observations &  \multicolumn{1}{c}{14,311} &  \multicolumn{1}{c}{14,311} \\
R-squared  &   \multicolumn{1}{c}{0.6141} & \multicolumn{1}{c}{0.7812} \\                                                                                                                     
\hline 
\hline \\[-1.8ex] 
\textit{Notes:}  & \multicolumn{2}{r}{$^{*}$p$<$0.1; $^{**}$p$<$0.05; $^{***}$p$<$0.01} \\ 
\end{tabular}
}
\caption{\tiny{Robust standard errors are clustered at the county level. Sample restricted to firms open in 1929}}
\end{table}
\end{frame}

\begin{frame}{}
\begin{itemize}
\item Industry-specific characteristics might have an important role to play in firm employment outcomes
\item Substantial variation exists in employment outcomes and revenue levels across the industries in the sample
\begin{itemize}
\item More capital intensive firms with higher dependence on external finance experienced larger decreases in revenue and employment over the Great Depression
\item Firms with greater dependence on external finance would be more adversely affected by bank distress than those that are not
\end{itemize}
\end{itemize}

\end{frame}

\begin{frame}{Variation in Labor Outcomes by Industry}
\begin{figure}
\centering{
\includegraphics[scale=.325]{"/Users/Adam/Research/BA_Thesis/Graphs/Average Change in Employment by Industry".pdf}
}
\end{figure}
\end{frame}

\begin{frame}{Variation in Revenue by Industry}
\begin{figure}
\centering{
\includegraphics[scale=.325]{"/Users/Adam/Research/BA_Thesis/Graphs/Average Change in Revenue by Industry".pdf}
}
\end{figure}
\end{frame}

\begin{frame}{Measuring Dependence on External Finance}
\begin{table}  \tiny{
\begin{tabular}{lc} 
\\[-1.8ex]\hline  
 \hline \\[-1.8ex] 
 \\[-1.8ex] \multicolumn{1}{c}{Industry in Sample} & \multicolumn{1}{c}{Dependence on External Finance} \\
  \\[-1.8ex] & \multicolumn{1}{c}{For Mature Companies} \\
 \hline \\[-1.8ex]                                                                             
 Automobiles & 0.14 \\  
 Aviation & 0.23 \\
 Blast Furnaces & 0.22 \\ 
 Bone Black & 0.05 \\
 Cane Sugar & -0.10 \\
 Cement & 0.15 \\
 Cigars and Cigarettes & -0.12 \\
Concrete & 0.15 \\
Glass & 0.03 \\
Ice & -0.10 \\
Ice Cream & -0.10 \\
Linoleum & 0.10 \\
Macaroni & -0.10 \\
Malt & -0.10 \\
Matches & -0.05 \\
Petroleum Refining & 0.07 \\
Radio & 0.39 \\
Rubber Tires & -0.12 \\
Soap & -0.05 \\
Steel Works & 0.07 \\
Sugar, Refining & -0.07 \\
Textiles & 0.14 \\
\hline 
\hline \\[-1.8ex] 
\end{tabular}
}
\caption{\tiny{Rajan and Zingales define a company's dependence on external finance to be its capital expenditures minus cash generated from operations, divided by total capital expenditures.}}
\end{table}
\end{frame}

\begin{frame}{Variation in Dependence on External Finance by Industry, continued}
\begin{itemize}
\item Create a dummy variable for firms in sample within an industry with high dependence on external finance $Ext\_Fin\_Dependence_i$ to see if effects if bank distress vary across two samples
\begin{itemize}
\item Firms deemed highly dependent if they're in an industry above the median level of external dependence
\item Modify the previous model by adding dependence dummy to current interaction terms:

\[
 Ext\_Fin\_Dependence_iXPost\_1929XBank\_Distress_c 
\]

\end{itemize}
\end{itemize}

\end{frame}

\begin{frame}{Dependence on External Finance -- Results}
\begin{table}  \tiny{
\begin{tabular}{lc} 
\\[-1.8ex]\hline  
 \hline \\[-1.8ex] 
 \\[-1.8ex] log(Labor) & \multicolumn{1}{c}{(1)} \\
 \hline \\[-1.8ex]                                                                             
 High Dependence on External Finance x Bank Distress & $-0.276^{***}$ \\ 
           & (0.091) \\ 
  & \\ 
 Post\_1929 x Bank Distress & $-0.038$ \\ 
  & (0.058) \\ 
  & \\ 
 High Dependence on External Finance x Post\_1929 x Bank Distress & $0.070$ \\ 
  & (0.108) \\ 
\hline \\[-1.8ex] 
Year Fixed Effects & \multicolumn{1}{c}{X} \\ 
County Fixed Effects  & \multicolumn{1}{c}{X} \\ 
Industry Fixed Effects &  \multicolumn{1}{c}{X} \\
Observations & \multicolumn{1}{c}{34,207}  \\
R-squared  &   \multicolumn{1}{c}{0.6923}  \\
\hline 
\hline \\[-1.8ex] 
\textit{Notes:}  & \multicolumn{1}{r}{$^{*}$p$<$0.1; $^{**}$p$<$0.05; $^{***}$p$<$0.01} \\ 
\end{tabular}
}
\caption{\tiny{Robust standard errors are clustered at the county level.}}
\end{table}
\end{frame}

\begin{frame}{Capital-Labor Substitution}
\begin{itemize}
\item There may be variation in how much industries substitute labor for capital during a financial crisis
\item An automobile plant is more capital-intensive than a textile mill, and might lay off more workers to maintain operating capital levels
\item Work in progress
\end{itemize}
\end{frame}


\end{document}