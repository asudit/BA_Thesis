\documentclass[letter,11pt]{article}
%\selectlanguage{english}

\AtBeginDocument{\setstretch{1.5}}

%%% PACKAGES %%%
\usepackage[english]{babel} % hyphenation
\usepackage[margin=2cm]{geometry} % margins
\usepackage{graphicx} % support for graphics
\usepackage{amsmath} % support for math­e­mat­i­cal typesetting
\usepackage{amssymb} % math­e­mat­i­cal symbols
\usepackage{dsfont}
\usepackage{color} % support for colors
\usepackage{mathtools} % more math­e­mat­i­cal type­set­ting
\usepackage{amsthm} % for defining theorem-like environments
\usepackage{enumerate} % change appearance of numbered lists
\usepackage{framed} % textboxes
\usepackage[format=plain,labelfont=bf,up]{caption} % cus­tomise cap­tions for fig­ures and ta­bles
\usepackage[colorlinks=true,linkcolor=black,urlcolor=blue,linktoc=all, citecolor=black]{hyperref} % hyperlinks
\usepackage[usenames,dvipsnames]{xcolor} % more colors
\usepackage{setspace} % set­ting the spac­ing be­tween lines
\usepackage{titlesec} % sec­tion­ing com­mands
\usepackage{subfigure} % multiple figures arranged neatly
\usepackage{natbib} % references
\usepackage{pdfpages} % import pages from other pdf fiels
\usepackage{comment} % comment out large sections
\usepackage{cancel} % strikethrough math
\usepackage{verbatim} % for including code
\usepackage{titlesec} % change font size of section
\usepackage{dcolumn}
\usepackage{pdfpages}
\usepackage{listings}

\usepackage{graphicx}
\graphicspath{ {images/} }

%%% CUSTOM COMMANDS %%%
\def\ci{\perp\!\!\!\perp} % statistical independence symbol
\def\nci{\perp\!\!\!\!\!\!\diagup\!\!\!\!\!\!\perp}
\newcommand{\ind}{1\hspace{-2.1mm}{1}} % indicator function
\newcommand{\rl}{\mathbb{R}} % real numbers
\newcommand{\ex}[1]{\mathbb{E} \left[ #1 \right]} % expectation operator
\newcommand{\pr}[1]{\mathbb{P} \left( #1 \right)} % probability
\newcommand{\var}[1]{\mathbb{V}\text{ar} \left[ #1 \right]} % variance
\newcommand{\cov}[1]{\mathbb{C}{ov} \left[ #1 \right]} % covariance
\newcommand{\corr}[1]{\mathbb{C}{orr} \left[ #1 \right]} % correlation
\newcommand{\tab}{\hspace{.4cm}}
\newcommand{\lra}{\text{ }\Leftrightarrow\text{ }}
\newcommand{\ra}{\text{ }\Rightarrow\text{ }}
\newcommand{\la}{\text{ }\Leftarrow\text{ }}
\newcommand{\veps}{\varepsilon}
\newcommand{\spc}{\text{ }}


%%% SET FONT %%%
\renewcommand{\familydefault}{\sfdefault}
\linespread{1.2}
\setlength{\parindent}{0pt}
\usepackage{sfmath}
\titleformat*{\section}{\normalsize\bfseries}
\titleformat*{\subsection}{\normalsize\bfseries}
\titleformat*{\subsubsection}{\normalsize\bfseries}
\titleformat*{\paragraph}{\normalsize\bfseries}
\titleformat*{\subparagraph}{\normalsize\bfseries}




\begin{document}

\title{\vskip-1cm \small The University of Chicago /  Winter 2016\footnote{This document was updated on \today.} \\
\Large Structural Model \vskip-.2cm}
\author{\normalsize Adam Sudit}
\date{February 2016}
\maketitle

\section{Motivation and Intuition}

I borrow heavily from Bernanke (1983) and Bernanke and Gilchrist(1995). The goal is to indentify as robustly as possible the lending channel between lenders (banks) and borrowers (manufacturing firms), in times of economic crisis. Consider a world of borrowers and lenders. In a world of assymmetric information, lenders such as banks provide a key function -- they undergo the costly process of identifying good borrowers, those who are likely to use loans productively and provide a return on investment greater than the cost of providing the loan, and bad borrowers, who do the opposite. Banks are the intermediaries that channel funds from investors or savers to good borrowers. The cost of lending to good borrowers, is known as the cost of credit indermediation (CCI) (Bernanke). Associated costs include auditing of potential borrowers, and conducting audits of those who have received loans. The cost of credit intermediation is a non-monetary channel through which panics can affect real economic activity.
\\ \\ 
Bernanke has decsribed two components of economic panics that can raise the CCI, making it difficult and costly for firms to obtain credit. The first is the effect of banking crisis on the CCI. This includes declining balance sheet health, bank runs on deposits, and bank suspensions. All of this leads to a decrease in the supply of credit. The second component is the effect of borrower bankrupticies on the CCI. Bernanke considers a world of incomplete markets, in which creating lending contracts that account for all possible future states of the world is costly and infeasible. Simple loan agreements are made, since they are less expensive. In the abscence of costly, complex loan agreements, collateral is an essential way in which banks lower the risk of making loans. Bernanke notes that the Great Depression from 1930-1933 can be thought of as the\"progressive erosion of borrowers collateral relative to debt burdens" (pg. 19 in PDF). When a borrower's collateral deteriorates in the event of economic collapse, the CCI increases, since the risk of a firm becoming insolvent increases. In many scenarios, banks will then 'fly to quality". While alternative forms of financing exist for a firm, it is still costly to secure loans from these sources of financing. Bernanke (1983) notes that during the greater depression, smaller households and companies had the most dependence on loans from banks (pg. 17 in PDF), and thus were the mostly adversely affeced. This will be important to our analysis later. 

\section{Model}
We need to a tractable way to impose financial constraints into the economy. Consider banks first. Based on the Bernanke-Blinder (1988) (Bernanke and Gilchrist 1995, pg 21 in PDF), we can reasonably assume that during the Great Depression, banks faced a constraint on how easily they could obtain new deposits. Based on Khwaja-Mian (2008), consider the following loan transaction. Assume for simplicity that a firm i receives loans from only one bank, bank j. Both the bank and firm live for 2 periods, and engage in a loan each period. In period t, the bank and the firm negociate a loan $L_{ij}^t$. To finance the loan, the bank raises deposits $D_i^t$, but as mentioned, it becomes costly to raise deposits past some level $D_i^t \leq \bar{D}$. Banks can also raise funds outside of deposits, $B_i^t$, but face a per unit cost of $\alpha_B$ in doing so. Lastly, assume that the share of bank i's funds that go into the loan for firm j is $\gamma_j$. So we have the following accounting equation:
\begin{align}
L_{ij}^t = \gamma_j (B_i^t + D_i^t)
\end{align}
In an incomplete market with assymetric information, there are transaction costs associated with making the loan. The bank must audit the firm's finances and assess the riskiness of the loan, and continue to monitor the firm as it makes loan repayments in the future. We can call this the cost of credit intermediation (CCI-Bernanke 1983), denoted as a per unit cost of $\alpha_L$. If a bank expects a return in period t+1 from the firm of $r_j^{t+1}$, then first order conditions tell us marginal cost of making a loan equals marginal benefit, and so we have the following identity:
\begin{align}
r_j^{t+1} = \gamma_j \alpha_B B_i^t + \alpha_L L_{ij}^t
\end{align}

We now turn to the firm. The firm has two sources of funding -- loans from bank j, $L_{ij}^t$, and external financing, $E_{ij}^t$. External financing requires connections with other investors, and higher levels of collateral. So suppose only large firms are able to find external financing, and then access to external financing by firm j is given by $\mathds{1}_{Large firm} E_{ij}^t$. If firm j receives a loan $L_{ij}^t$, and has output from earlier in period t of $Y_j^t$ (firms produce at beginning of each period), then they the following budget constraint going into period t+1:
\begin{align}
L_{ij}^t + \mathds{1}_{Large firm} E_{ij}^t + Y_j^t = K_j^{t+1} + L_j^{t+1} + r_j^{t+1}
\end{align}
Lastly, for tractibility, define the following relationship between the collateral of firm j at the beginning of period t, $Y_j^t$ and the cost of credit intermediation, $\alpha_L$: $\frac{1}{Y_j^t} = \alpha_L$. While a little contrived, the identity is consistent with the inutution developed in section 1. Since markets are incomplete, it is infeasible to create complicated loan agreements. Instead simple loan agreements are made that do not account for all possible states of the world. The key mechanism in which banks can lower the risk of the transaction is through collateral. If a firm's collateral in the beginning of period t $Y_j^t$, after production but before the loan agreement is large enough, the costliness of the transaction is less, since the implied risk of the loan is small. Converesly, a small amount of collateral makes the loan very risky, making it prohibitively expensive to make the loan.

\end{document}